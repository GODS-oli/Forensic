\documentclass[12pt, a4paper]{report}
\usepackage[utf8]{inputenc}
\usepackage[T1]{fontenc}
\usepackage[french]{babel}
\usepackage[top=0.6in, bottom=0.6in, right=1in, left=1in]{geometry}
\usepackage{amsmath, amssymb}
\usepackage{graphicx, xcolor}
\usepackage{booktabs, multirow, tabularx}
\usepackage{hyperref}
\usepackage{paralist, enumitem}
\usepackage{colortbl, float, caption, subcaption}
\usepackage{mwe}

\setlist[itemize]{leftmargin=*}
\setlist[enumerate]{leftmargin=*}
\definecolor{blue}{RGB}{31,56,100}

\title{\textbf{Résumé du Cours d’Introduction à l’Investigation Numérique}}
\author{Olivia GHOUMO}
\date{Septembre 2025}

\begin{document}

\begin{titlepage}
\begin{center}
    \begin{minipage}{0.2\textwidth}
        \centering
        \includegraphics[height=1.8cm]{ENSPY.png}
    \end{minipage}\hfill
    \begin{minipage}{0.6\textwidth}
        \centering
        \textbf{ECOLE NATIONALE SUPERIEURE POLYTECHNIQUE DE YAOUNDE}\\[0.1cm]
        \textbf{DEPARTEMENT DES ARTS ET HUMANITES NUMERIQUES}\\[0.1cm]
    \end{minipage}\hfill
    \begin{minipage}{0.2\textwidth}
        \centering
        \includegraphics[height=1.8cm]{ENSPY.png}
    \end{minipage}

    \vspace{2.5cm}
    \rule{\linewidth}{0.3mm} \\[0.4cm]
    {\huge \bfseries\color{blue} RESUME DU COURS D'INTRODUCTION À L'INVESTIGATION NUMÉRIQUE \\[0.4cm] }
    \rule{\linewidth}{0.3mm} \\[3cm]

    \noindent
    \begin{minipage}{0.4\textwidth}
        \begin{flushleft} \large
            \emph{Présenté par :}\\
            \textbf{GHOUMO DONFACK}\\
            \textbf{Olivia Shelsie}\\
            \textsc{Classe :} CIN 04\\
            \textsc{Matricule :} 22p023\\
            \textsc{Professeur :} M. Thierry MINKA
        \end{flushleft}
    \end{minipage}
    \vfill

    {\textbf{\large Année universitaire 2025-2026}}
\end{center}
\end{titlepage}

\newpage
\section*{1. Introduction et Philosophie de l’Investigation Numérique}
\subsection*{1.1 Contexte et Enjeux}
\begin{itemize}
    \item L’investigation numérique est une discipline \textbf{philosophique, technique et juridique}.
    \item Elle interroge la nature de la \textbf{vérité}, de la \textbf{confiance} et de la \textbf{justice} à l’ère numérique.
    \item \textbf{Problématique} : Comment concilier \textbf{confidentialité}, \textbf{fiabilité} et \textbf{opposabilité juridique} des preuves numériques dans un contexte \textbf{post-quantique} ?
\end{itemize}

\subsection*{1.2 Le Paradoxe de l’Authenticité Invisible}
\begin{itemize}
    \item Plus une preuve est authentique et vérifiable, plus elle risque de compromettre la confidentialité (et inversement).
    \[
    \Delta A \cdot \Delta C \geq h_{num}
    \]
    \begin{itemize}
        \item \(\Delta A\) : Incertitude sur l’authenticité.
        \item \(\Delta C\) : Incertitude sur la confidentialité.
        \item \(h_{num}\) : Constante numérique fondamentale.
    \end{itemize}
\end{itemize}

\subsection*{1.3 Le Trilemme CRO}
\begin{itemize}
    \item \textbf{C}onfidentialité : Protection des données sensibles.
    \item \textbf{R}eliabilité (Fiabilité) : Intégrité et authenticité des preuves.
    \item \textbf{O}pposabilité juridique : Valeur probante en justice.
    \item Il est théoriquement impossible de maximiser ces trois points.
\end{itemize}

\subsection*{1.4 Éthique et Responsabilité}
\begin{itemize}
    \item \textbf{Serment de l’Investigateur Numérique} :
    \begin{itemize}
        \item Utiliser les connaissances exclusivement à des fins légitimes.
        \item Respecter les cadres juridiques nationaux et internationaux.
        \item Protéger la confidentialité et l’\textbf{intégrité des systèmes}.
        \item Documenter intégralement les méthodologies.
    \end{itemize}
    \item \textbf{Quatre piliers déontologiques} :
    \begin{enumerate}
        \item \textbf{Intégrité} (véracité, transparence).
        \item \textbf{Proportionalité} (adéquation des moyens).
        \item \textbf{Responsabilité} (devoir de vigilance).
        \item \textbf{Service} (mettre les compétences au service de la justice).
    \end{enumerate}
\end{itemize}

\section*{2. Cadre Théorique et Conceptuel}
\subsection*{2.1 Fondements Théoriques}
\begin{itemize}
    \item \textbf{Théorie de l’Information (Shannon)} :
    \begin{itemize}
        \item \textbf{Entropie} pour détecter des anomalies.
        \item \textbf{Distance de Hamming} pour analyser la similarité entre fichiers.
    \end{itemize}
    \item \textbf{Théorie des Graphes} :
    \begin{itemize}
        \item Modélisation des \textbf{réseaux sociaux} et des \textbf{flux de données}.
        \item Détection de \textbf{communautés cachées} et de \textbf{chemins de fuite}.
    \end{itemize}
    \item \textbf{Théorie du Chaos} :
    \begin{itemize}
        \item Sensibilité aux \textbf{conditions initiales}.
    \end{itemize}
\end{itemize}

\subsection*{2.2 Modèles d’Investigation}
\begin{table}[h]
    \centering
    \caption{Comparaison des Modèles d’Investigation}
    \begin{tabularx}{\textwidth}{|>{\bfseries}l|X|X|}
        \hline
        Modèle & Phases Clés & Application Typique \\ \hline
        DFRWS (2001) &
        Identification $\rightarrow$ Préservation $\rightarrow$ Collecte $\rightarrow$ Examen $\rightarrow$ Analyse $\rightarrow$ Présentation &
        Standard académique et opérationnel. \\ \hline
        Casey (2004) &
        Préparation $\rightarrow$ Déploiement $\rightarrow$ Scène physique $\rightarrow$ Scène numérique $\rightarrow$ Révision &
        Enquêtes criminelles complexes. \\ \hline
        ISO/IEC 27037 &
        Identification $\rightarrow$ Collecte $\rightarrow$ Préservation $\rightarrow$ Documentation &
        Norme internationale pour la collecte. \\ \hline
        NIST SP 800-86 &
        Collection $\rightarrow$ Examen $\rightarrow$ Analyse $\rightarrow$ Rapport &
        Intégration à la réponse aux incidents. \\ \hline
    \end{tabularx}
\end{table}

\subsection*{2.3 Normes et Standards Internationaux}
\begin{itemize}
    \item \textbf{ISO/IEC 27037} : Lignes directrices pour l’identification, la collecte et la préservation des preuves numériques.
    \item \textbf{NIST SP 800-86} : Intégration des techniques forensiques à la réponse aux incidents.
    \item \textbf{RFC 3227} : Ordre de volatilité (Farmer \& Venema) pour la collecte des preuves.
    \item \textbf{ACPO (Royaume-Uni)} : 4 principes clés (pas de modification des données, compétence requise, audit trail, responsabilité).
\end{itemize}

\section*{3. L’Ère Post-Quantique et ses Défis}
\subsection*{3.1 Menaces Quantiques}
\begin{table}[h]
    \centering
    \caption{Impact des Algorithmes Quantiques sur la Cryptographie Classique}
    \begin{tabularx}{\textwidth}{|l|X|X|}
        \hline
        \textbf{Algorithme Quantique} & \textbf{Impact sur la Cryptographie Classique} & \textbf{Conséquences pour l’Investigation} \\ \hline
        Shor & Cassage de RSA et ECC en temps polynomial. & Preuves historiques compromises. \\ \hline
        Grover & Réduction de moitié de la sécurité des clés symétriques. & Attaques "Harvest Now, Decrypt Later". \\ \hline
    \end{tabularx}
\end{table}

\subsection*{3.2 Cryptographie Post-Quantique (PQC)}
\begin{itemize}
    \item \textbf{Algorithmes sélectionnés par le NIST (2022)} :
    \begin{itemize}
        \item \textbf{Signatures} : CRYSTALS-Dilithium (basé sur les réseaux), SPHINCS+ (hash-based).
        \item \textbf{Chiffrement} : CRYSTALS-Kyber (KEM basé sur LWE).
    \end{itemize}
    \item \textbf{Migration progressive} :
    \begin{itemize}
        \item \textbf{Court terme} : Hybridation (RSA + Kyber).
        \item \textbf{Long terme} : Remplacement complet par PQC.
    \end{itemize}
\end{itemize}

\subsection*{3.3 Quantum Forensics}
\begin{itemize}
    \item Analyse de nombres aléatoires quantiques (QRNG vs PRNG).
    \item Tomographie d’état quantique pour valider l’intégrité des preuves.
    \item Protocoles ZK-NR pour une non-répudiation post-quantique.
\end{itemize}

\section*{4. Le Protocole ZK-NR : Innovation Majeure}
\subsection*{4.1 Architecture du Protocole}
\begin{itemize}
    \item \textbf{Couches} :
    \begin{enumerate}
        \item \textbf{Merkle Commitments} : Structure d’engagement pour l’intégrité.
        \item \textbf{STARK Proofs} : Preuves zero-knowledge post-quantiques et transparentes.
        \item \textbf{Threshold BLS} : Signatures distribuées pour la résilience.
        \item \textbf{Dilithium} : Authentication post-quantique.
    \end{enumerate}
    \item \textbf{Flux} :
    \begin{itemize}
        \item Engagement $\rightarrow$ Preuve ZK $\rightarrow$ Signature à seuil $\rightarrow$ Authentification PQC.
    \end{itemize}
\end{itemize}

\subsection*{4.2 Sécurité UC (Universal Composability)}
\begin{itemize}
    \item \textbf{Modèle de sécurité} :
    \begin{itemize}
        \item Non-répudiation.
        \item Zero-Knowledge (aucune information révélée).
        \item Résistance quantique.
    \end{itemize}
    \item \textbf{Preuve formelle} (via Tamarin Prover) :
    \begin{itemize}
        \item Vérification des propriétés de confidentialité, intégrité et authenticité.
    \end{itemize}
\end{itemize}

\subsection*{4.3 Applications Pratiques}
\begin{itemize}
    \item \textbf{Chaîne de custody post-quantique} :
    \begin{itemize}
        \item Horodatage quantique + signatures ZK-NR pour une traçabilité inviolable.
    \end{itemize}
    \item \textbf{Preuves judiciaires} :
    \begin{itemize}
        \item Opposabilité garantie même dans un contexte quantique.
    \end{itemize}
\end{itemize}

\section*{5. Techniques d’Anti-Anti-Forensique}
\begin{itemize}
    \item \textbf{Contournement de chiffrement} :
    \begin{itemize}
        \item Cold Boot Attack (récupération de clés en RAM refroidie).
        \item Evil Maid Attack (installation de keyloggers hardware).
    \end{itemize}
    \item \textbf{Détection de stéganographie} :
    \begin{itemize}
        \item Analyse statistique (entropie, chi-square).
        \item Machine Learning pour identifier les patterns cachés.
    \end{itemize}
    \item \textbf{Déobfuscation de code} :
    \begin{itemize}
        \item Analyse dynamique (sandboxing).
        \item Symbolic Execution pour comprendre les logiques obscurcies.
    \end{itemize}
\end{itemize}

\section*{6. Cadre Juridique et Applications Pratiques}
\subsection*{6.1 Législation Mondiale}
\begin{table}[h]
    \centering
    \caption{Cadre Juridique par Région}
    \begin{tabular}{|l|l|}
        \hline
        \textbf{Région} & \textbf{Cadre Juridique Clé} \\ \hline
        États-Unis & FRE (Federal Rules of Evidence), CFAA (Computer Fraud and Abuse Act). \\ \hline
        Europe & eIDAS (signatures électroniques), RGPD (protection des données), Convention de Budapest. \\ \hline
        Afrique & Convention de Malabo (cybercriminalité), Loi camerounaise 2010/012. \\ \hline
        Cameroun & Loi 2010/012 (cybersécurité), Loi 2024/017 (protection des données). \\ \hline
    \end{tabular}
\end{table}

\subsection*{6.2 Procédure d’Investigation au Cameroun}
\begin{enumerate}
    \item Plainte/Signalement.
    \item Enquête préliminaire.
    \item Commission rogatoire.
    \item Expertise judiciaire.
    \item Rapport d’expertise.
    \item Audience.
\end{enumerate}

\subsection*{6.3 Cas Pratique : Affaire CyberFinance Cameroun 2025}
\begin{itemize}
    \item \textbf{Scénario} :
    \begin{itemize}
        \item Attaque ransomware (LockBit 3.0) sur une fintech camerounaise.
        \item Exfiltration de 850 GB de données clients.
        \item Demande de rançon : 10M EUR en Bitcoin.
    \end{itemize}
    \item \textbf{Réponse} :
    \begin{itemize}
        \item Isolation du réseau $\rightarrow$ Acquisition forensique (ISO 27037) $\rightarrow$ Analyse ZK-NR.
        \item Attribution : Groupe LockBit affiliate "GoldManager" (Europe de l’Est).
        \item Remédiation : Migration vers CRYSTALS-Kyber et Dilithium, déploiement du framework Q2CSI.
    \end{itemize}
\end{itemize}

\newpage
\section*{7. Benchmarking Mondial et Best Practices}
\subsection*{7.1 Comparaison des Approches}
\begin{table}[h]
    \centering
    \caption{Benchmarking des Agences Forensiques Mondiales}
    \begin{tabular}{|l|l|l|l|}
        \hline
        \textbf{Agence} & \textbf{Forces} & \textbf{Faiblesses} & \textbf{Score CRO} \\ \hline
        FBI/NIST & Innovation technologique, normalisation. & Complexité juridique. & 9.13 \\ \hline
        Scotland Yard & Rigueur procédurale, coopération internationale. & Lenteur administrative. & 8.77 \\ \hline
        BKA (Allemagne) & Précision technique, validation métrologique. & Manque de flexibilité. & 8.78 \\ \hline
        Singapour & Technologie de pointe, IA et IoT. & Coût élevé. & 8.71 \\ \hline
        France (ANSSI) & Souveraineté numérique, cadre juridique solide. & Adoption lente des innovations. & 8.16 \\ \hline
    \end{tabular}
\end{table}

\subsection*{7.2 Framework d’Excellence Universelle}
\begin{itemize}
    \item \textbf{Hybridation des meilleures pratiques} :
    \begin{itemize}
        \item Innovation américaine (FBI/NIST) + rigueur allemande (BKA) + adaptabilité africaine.
    \end{itemize}
    \item \textbf{Recommandations stratégiques} :
    \begin{enumerate}
        \item Former les investigateurs aux protocoles ZK-NR et Q2CSI.
        \item Migrer vers PQC (Kyber, Dilithium) d’ici 2030.
        \item Automatiser la chaîne de custody avec des preuves cryptographiques.
        \item Renforcer la coopération internationale (Convention de Budapest, MLAT).
    \end{enumerate}
\end{itemize}

\section*{8. Synthèse et Perspectives}
\subsection*{8.1 Leçons Clés}
\begin{enumerate}
    \item Le Trilemme CRO est inévitable : Aucune solution ne maximise simultanément confidentialité, fiabilité et opposabilité.
    \item Les protocoles ZK-NR et Q2CSI offrent un équilibre optimal pour l’ère post-quantique.
    \item L’investigation numérique moderne doit intégrer :
    \begin{itemize}
        \item Cryptographie post-quantique.
        \item Intelligence artificielle (détection d’anomalies, attribution).
        \item Cadre juridique adaptatif.
    \end{itemize}
\end{enumerate}

\subsection*{8.2 Roadmap pour l’Afrique}
\begin{table}[h]
    \centering
    \caption{Roadmap pour l’Afrique}
    \begin{tabular}{|l|l|}
        \hline
        \textbf{Horizon} & \textbf{Actions Clés} \\ \hline
        2025-2027 & Formation aux protocoles ZK-NR, déploiement de laboratoires forensiques. \\ \hline
        2027-2030 & Migration vers PQC (Kyber, Dilithium), adoption du framework Q2CSI. \\ \hline
        2030+ & Leadership mondial en investigation post-quantique (leapfrogging). \\ \hline
    \end{tabular}
\end{table}

\begin{quote}
« L’investigation numérique n’est pas qu’une discipline technique, mais une \textbf{praxis de liberté} : elle protège la vérité dans un monde où le numérique redéfinit constamment les frontières du réel. »
— \textbf{MalEtYOn}
\end{quote}

\end{document}
